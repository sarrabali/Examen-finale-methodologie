\documentclass[12pt,a4paper]{article}

% --- PACKAGES DE BASE ---
\usepackage[hyphens]{url}
\usepackage[utf8]{inputenc}
\usepackage[T1]{fontenc}
\usepackage[french]{babel}
\usepackage{lmodern}
\usepackage{amsmath,amsfonts,amssymb}
\usepackage{algorithm,algpseudocode}
\usepackage{graphicx}
\usepackage{tabularx}
\usepackage{booktabs}
\usepackage{geometry}
\usepackage{csquotes}
\usepackage{mwe} % fournit example-image-a pour la figure d'exemple

\graphicspath{{figures/}}

\newtheorem{theorem}{Théorème}

% --- MARGE & PDF ---
\geometry{margin=2.5cm}

% --- Algorithmic francais, voir https://tex.stackexchange.com/a/438815 ---
\renewcommand{\listalgorithmname}{Liste des algorithmes}
\floatname{algorithm}{Algorithme}
\renewcommand{\algorithmicreturn}{\textbf{retourne}}
\renewcommand{\algorithmicprocedure}{\textbf{procédure}}
%\renewcommand{\Not}{\textbf{non}\ }
\renewcommand{\And}{\textbf{et}\ }
%\renewcommand{\Or}{\textbf{ou}\ }
\renewcommand{\algorithmicrequire}{\textbf{Entrée:}}
\renewcommand{\algorithmicensure}{\textbf{Sortie:}}
%\renewcommand{\algorithmiccomment}[1]{\{#1\}}
\renewcommand{\algorithmicend}{\textbf{fin}}
\renewcommand{\algorithmicif}{\textbf{si}}
\renewcommand{\algorithmicthen}{\textbf{alors}}
\renewcommand{\algorithmicelse}{\textbf{sinon}}
\renewcommand{\algorithmicfor}{\textbf{pour}}
\renewcommand{\algorithmicforall}{\textbf{pour tout}}
\renewcommand{\algorithmicdo}{\textbf{faire}}
\renewcommand{\algorithmicwhile}{\textbf{tant que}}
\renewcommand{\algorithmicfunction}{\textbf{fonction}}
\newcommand{\algorithmicelsif}{\algorithmicelse\ \algorithmicif}
\newcommand{\algorithmicendif}{\algorithmicend\ \algorithmicif}
\newcommand{\algorithmicendfor}{\algorithmicend\ \algorithmicfor}

% --- HYPERREF EN DERNIER ---
\usepackage{hyperref}
\hypersetup{
	colorlinks=true,
	linkcolor=blue,
	citecolor=blue,
	urlcolor=blue
}

% --- MÉTADONNÉES DU DOCUMENT ---
\title{INF5163 -- Méthodologie de recherche en informatique\\[0.5em]
	\textbf{Examen Finale, Groupe 8}\\[0.5em]
	\emph{De la méthode à l’intégrité : gouvernance éthique et légale d’un outil d’IA
générative pour la recherche (\emph{RedactoSci})}}

\author{Paulin Rodrigue Njayou Tchapda, Sarra Yasmine Bali\\
	Rachidatou Mabey Insa et Fatimata Zahra Diop\\[0.5em]
	\emph{Université du Québec en Outaouais}}
\date{25 Novembre, Session Automne 2025}

\begin{document}

% 1) Page de titre
\maketitle

% 2) Table des matières
\tableofcontents
\listoftables
\listoffigures

\newpage
\section{Introduction et justification du sujet}
\label{sec:Introduction et justification du sujet}


\section{Problématique et objectifs}
\label{sec:Problématique et objectifs}

\section{Rappel du schéma méthodologique}
\label{sec:Rappel du schéma méthodologique}

\section{Analyse d’intégrité, éthique et ÉDI}
\label{sec:Analyse d’intégrité, éthique et ÉDI}

\section{Gestion et propriété intellectuelle}
\label{sec:Gestion et propriété intellectuelle}

\section{Valorisation et Diffusion}
\label{sec:Valorisation et Diffusion}
\section{Valorisation et Diffusion}
\label{sec:Valorisation et Diffusion}

\subsection{Plan de valorisation}

\subsubsection{Option 1 : Création d'une spin-off universitaire (commercialisation de RedactoSci)}

Cette approche vise la création de valeur économique et le transfert de technologie vers le marché.

\textbf{Avantages :}
\begin{itemize}
    \item \textbf{Génération de revenus :} Potentiel de profits pour l'université, les chercheurs impliqués et la spin-off.
    \item \textbf{Focalisation stratégique :} La spin-off, en tant qu'entité indépendante, peut se concentrer sur le développement, la maintenance et la commercialisation de l'outil.
    \item \textbf{Impact économique :} Contribution au développement économique local (emplois, innovation).
    \item \textbf{Accès facilité au financement :} Attire les investissements privés (capital-risque) pour accélérer la croissance.
\end{itemize}

\textbf{Inconvénients :}
\begin{itemize}
    \item \textbf{Barrières à l'accès :} Le coût peut limiter l'utilisation de l'outil par la communauté scientifique ou le grand public.
    \item \textbf{Risque entrepreneurial :} La majorité des spin-offs universitaires rencontrent des difficultés à atteindre un succès rapide.
    \item \textbf{Conflit d’intérêts :} Potentiel de tensions entre culture académique (partage) et culture d’entreprise (confidentialité, profit).
\end{itemize}

\subsubsection{Option 2 : Mise à disposition gratuite comme service public (attractivité et impact)}

Cette approche met l'accent sur l'impact sociétal, la visibilité académique et le rayonnement institutionnel.

\textbf{Avantages :}
\begin{itemize}
    \item \textbf{Attraction des talents :} Un outil libre augmente la visibilité du laboratoire et attire des profils d'excellence.
    \item \textbf{Impact sociétal maximal :} Favorise une adoption large dans la recherche, l’éducation ou la société.
    \item \textbf{Renforcement de la réputation :} Valorise l'université comme acteur de l’innovation ouverte.
    \item \textbf{Accès aux financements publics :} Facilite l’obtention de subventions axées sur l’intérêt général.
\end{itemize}

\textbf{Inconvénients :}
\begin{itemize}
    \item \textbf{Coûts de maintenance :} Nécessite un financement continu pour assurer la durabilité du service.
    \item \textbf{Manque d’agilité commerciale :} Les structures publiques sont parfois moins flexibles face aux besoins du marché.
    \item \textbf{Difficulté à capter la valeur économique :} Le retour financier direct pour les créateurs est limité.
\end{itemize}

\subsubsection{Plan de valorisation recommandé : Modèle hybride}

Une combinaison des deux approches permet de concilier impact scientifique et viabilité économique.

\begin{itemize}
    \item \textbf{Accès Freemium :} Version gratuite de base pour la communauté académique afin de maximiser l’adoption et la visibilité.
    \item \textbf{Spin-off pour services Premium :} Commercialisation de fonctionnalités avancées, support technique, licences professionnelles et services de consulting.
    \item \textbf{Stratégie de propriété intellectuelle :} Protection par brevets et licences spécifiques pour un usage académique.
\end{itemize}

Ce modèle assure simultanément visibilité, impact scientifique et potentiel économique.

\subsection{Plan de diffusion}

Deux publics clés sont identifiés pour la diffusion scientifique et institutionnelle.

\subsubsection{Pour les éditeurs scientifiques}

\begin{itemize}
    \item \textbf{Adapter l’article à la revue :} Sélection de revues pertinentes (IEEE, Springer) selon le domaine et respect de leurs exigences.
    \item \textbf{Mettre en avant l’innovation :} Souligner le caractère novateur et l’impact de RedactoSci.
    \item \textbf{Utiliser un langage scientifique rigoureux :} Présentation factuelle, claire et structurée.
    \item \textbf{Valoriser les données :} Appuyer les conclusions par des résultats solides et reproductibles.
\end{itemize}

\subsubsection{Pour les comités d’éthique de la recherche}

\begin{itemize}
    \item \textbf{Respect des principes éthiques :} Protection des données, confidentialité, consentement éclairé.
    \item \textbf{Description des procédures :} Explication claire du processus de consentement, durée de participation et gestion des risques.
    \item \textbf{Fourniture de documents de soutien :} Formulaires, protocoles, procédures de protection des données.
    \item \textbf{Valorisation des bénéfices sociétaux :} Mise en lumière des retombées scientifiques, technologiques ou éducatives.
\end{itemize}
 



\section{Contributions individuelles}
\label{sec:Contributions individuelles}

\newpage
% 5) Bibliographie via BibTeX


\newpage
\pagenumbering{roman}
\appendix



\end{document}
