\documentclass[12pt,a4paper]{article}

% --- PACKAGES DE BASE ---
\usepackage[utf8]{inputenc} % si vous compilez avec pdflatex. Inutile avec XeLaTeX ou LuaLaTeX.
\usepackage[T1]{fontenc}
\usepackage[french]{babel}
\usepackage{lmodern}
\usepackage{amsmath,amsfonts,amssymb}
\usepackage{algorithm,algpseudocode}
\usepackage{graphicx}
\usepackage{tabularx}
\usepackage{booktabs}
\usepackage{geometry}
\usepackage{csquotes}
\usepackage{hyperref}
\usepackage{mwe} % fournit example-image-a

\graphicspath{{figures/}}

\newtheorem{theorem}{Théorème}

% --- MARGE ---
\geometry{margin=2.5cm}

% --- PERSONNALISATION ALGORITHMES ---
\floatname{algorithm}{Algorithme}
\renewcommand{\listalgorithmname}{Liste des algorithmes}
\renewcommand{\algorithmicreturn}{\textbf{retourne}}
\renewcommand{\algorithmicprocedure}{\textbf{procédure}}
\renewcommand{\And}{\textbf{et}\ }
\renewcommand{\algorithmicrequire}{\textbf{Entrée:}}
\renewcommand{\algorithmicensure}{\textbf{Sortie:}}
\renewcommand{\algorithmicend}{\textbf{fin}}
\renewcommand{\algorithmicif}{\textbf{si}}
\renewcommand{\algorithmicthen}{\textbf{alors}}
\renewcommand{\algorithmicelse}{\textbf{sinon}}
\renewcommand{\algorithmicfor}{\textbf{pour}}
\renewcommand{\algorithmicforall}{\textbf{pour tout}}
\renewcommand{\algorithmicdo}{\textbf{faire}}
\renewcommand{\algorithmicwhile}{\textbf{tant que}}
\renewcommand{\algorithmicfunction}{\textbf{fonction}}
\newcommand{\algorithmicelsif}{\algorithmicelse\ \algorithmicif}
\newcommand{\algorithmicendif}{\algorithmicend\ \algorithmicif}
\newcommand{\algorithmicendfor}{\algorithmicend\ \algorithmicfor}

% --- HYPERLIENS ---
\hypersetup{
	colorlinks=true,
	linkcolor=blue,
	citecolor=blue,
	urlcolor=blue
}

% --- MÉTADONNÉES ---
\title{INF5163 -- Méthodologie de recherche en informatique\\[0.5em]
	\textbf{Examen Final, Groupe 8}\\[0.5em]
	\emph{De la méthode à l’intégrité : gouvernance éthique et légale d’un outil d’IA générative pour la recherche (\emph{RedactoSci})}}

\author{Paulin Rodrigue Njayou Tchapda, Sarra Yasmine Bali\\
	Rachidatou Mabey Insa et Fatimata Zahra Diop\\[0.5em]
	\emph{Université du Québec en Outaouais}}
\date{25 Novembre, Session Automne 2025}

\begin{document}

\maketitle

\tableofcontents
\listoftables
\listoffigures

\newpage

\section{Introduction et justification du sujet}
\label{sec:Introduction et justification du sujet}

\section{Problématique et objectifs}
\label{sec:Problématique et objectifs}

\section{Rappel du schéma méthodologique}
\label{sec:Rappel du schéma méthodologique}

\section{Analyse d’intégrité, éthique et ÉDI}
\label{sec:Analyse d’intégrité, éthique et ÉDI}

\section{Gestion et propriété intellectuelle}
\label{sec:Gestion et propriété intellectuelle}

% =================== SECTION VALORISATION ===================
\section{Valorisation et Diffusion}
\label{sec:Valorisation et Diffusion}

\subsection{Plan de valorisation}

\subsubsection{Option 1 : Création d'une spin-off universitaire (commercialisation de RedactoSci)}

Cette approche vise la création de valeur économique et le transfert de technologie vers le marché.

\textbf{Avantages :}
\begin{itemize}
    \item Potentiel de profits pour l'université, les chercheurs impliqués et la spin-off.
    \item Focalisation stratégique sur le développement, la maintenance et la commercialisation.
    \item Impact économique : création d’emplois, innovation locale.
    \item Accès facilité au financement (capital-risque, subventions privées).
\end{itemize}

\textbf{Inconvénients :}
\begin{itemize}
    \item Accessibilité limitée (coûts d'utilisation).
    \item Risque entrepreneurial d’échec.
    \item Conflits potentiels entre culture académique et logique commerciale.
\end{itemize}

\subsubsection{Option 2 : Service public gratuit (impact et attractivité)}

\textbf{Avantages :}
\begin{itemize}
    \item Forte visibilité académique et institutionnelle.
    \item Adoption large favorisant l’impact sociétal.
    \item Renforce la réputation de l'université en innovation ouverte.
    \item Facilite l’accès aux subventions publiques.
\end{itemize}

\textbf{Inconvénients :}
\begin{itemize}
    \item Financement et maintenance continus requis.
    \item Moins d’agilité commerciale.
    \item Difficile de capter la valeur économique directe.
\end{itemize}

\subsubsection{Modèle hybride recommandé}

\begin{itemize}
    \item \textbf{Accès Freemium} pour favoriser l’adoption académique.
    \item \textbf{Spin-off commerciale} pour les versions avancées et support professionnel.
    \item \textbf{Gestion de la PI :} licences, brevets et droits d’usage différenciés.
\end{itemize}

\subsection{Plan de diffusion}

\subsubsection{Pour les éditeurs scientifiques}
\begin{itemize}
    \item Adapter l’article à la revue (IEEE, Springer).
    \item Mettre en avant l’innovation scientifique et l’impact de l’outil.
    \item Utiliser un style rigoureux, structuré et factuel.
    \item Appuyer les résultats par des données solides et reproductibles.
\end{itemize}

\subsubsection{Pour les comités d’éthique}
\begin{itemize}
    \item Respect des principes éthiques : consentement, confidentialité, données.
    \item Décrire les procédures, risques, durée d’implication.
    \item Fournir documents : formulaires, protocoles, procédures.
    \item Mettre en valeur les bénéfices pour la société et la recherche.
\end{itemize}

% ================= END SECTION VALORISATION =================

\section{Contributions individuelles}
\label{sec:Contributions individuelles}

\newpage
% Bibliographie (BibTeX)

\newpage
\pagenumbering{roman}
\appendix

\end{document}
