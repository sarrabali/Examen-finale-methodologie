\documentclass[12pt,a4paper]{article}

% --- PACKAGES DE BASE ---
\usepackage[hyphens]{url}
\usepackage[utf8]{inputenc}
\usepackage[T1]{fontenc}
\usepackage[french]{babel}
\usepackage{lmodern}
\usepackage{amsmath,amsfonts,amssymb}
\usepackage{algorithm,algpseudocode}
\usepackage{graphicx}
\usepackage{tabularx}
\usepackage{booktabs}
\usepackage{geometry}
\usepackage{csquotes}
\usepackage{mwe} % fournit example-image-a pour la figure d'exemple

\graphicspath{{figures/}}

\newtheorem{theorem}{Théorème}

% --- MARGE & PDF ---
\geometry{margin=2.5cm}

% --- Algorithmic francais, voir https://tex.stackexchange.com/a/438815 ---
\renewcommand{\listalgorithmname}{Liste des algorithmes}
\floatname{algorithm}{Algorithme}
\renewcommand{\algorithmicreturn}{\textbf{retourne}}
\renewcommand{\algorithmicprocedure}{\textbf{procédure}}
%\renewcommand{\Not}{\textbf{non}\ }
\renewcommand{\And}{\textbf{et}\ }
%\renewcommand{\Or}{\textbf{ou}\ }
\renewcommand{\algorithmicrequire}{\textbf{Entrée:}}
\renewcommand{\algorithmicensure}{\textbf{Sortie:}}
%\renewcommand{\algorithmiccomment}[1]{\{#1\}}
\renewcommand{\algorithmicend}{\textbf{fin}}
\renewcommand{\algorithmicif}{\textbf{si}}
\renewcommand{\algorithmicthen}{\textbf{alors}}
\renewcommand{\algorithmicelse}{\textbf{sinon}}
\renewcommand{\algorithmicfor}{\textbf{pour}}
\renewcommand{\algorithmicforall}{\textbf{pour tout}}
\renewcommand{\algorithmicdo}{\textbf{faire}}
\renewcommand{\algorithmicwhile}{\textbf{tant que}}
\renewcommand{\algorithmicfunction}{\textbf{fonction}}
\newcommand{\algorithmicelsif}{\algorithmicelse\ \algorithmicif}
\newcommand{\algorithmicendif}{\algorithmicend\ \algorithmicif}
\newcommand{\algorithmicendfor}{\algorithmicend\ \algorithmicfor}

% --- HYPERREF EN DERNIER ---
\usepackage{hyperref}
\hypersetup{
	colorlinks=true,
	linkcolor=blue,
	citecolor=blue,
	urlcolor=blue
}

% --- MÉTADONNÉES DU DOCUMENT ---
\title{INF5163 -- Méthodologie de recherche en informatique\\[0.5em]
	\textbf{Examen Finale, Groupe 8}\\[0.5em]
	\emph{De la méthode à l’intégrité : gouvernance éthique et légale d’un outil d’IA
générative pour la recherche (\emph{RedactoSci})}}

\author{Paulin Rodrigue Njayou Tchapda, Sarra Yasmine Bali\\
	Rachidatou Mabey Insa et Fatimata Zahra Diop\\[0.5em]
	\emph{Université du Québec en Outaouais}}
\date{25 Novembre, Session Automne 2025}

\begin{document}

% 1) Page de titre
\maketitle

% 2) Table des matières
\tableofcontents
\listoftables
\listoffigures

\newpage
\section{Introduction et justification du sujet}
\label{sec:Introduction et justification du sujet}


\section{Problématique et objectifs}
\label{sec:Problématique et objectifs}

\section{Rappel du schéma méthodologique}
\label{sec:Rappel du schéma méthodologique}

\section{Analyse d’intégrité, éthique et ÉDI}
\label{sec:Analyse d’intégrité, éthique et ÉDI}

\section{Gestion et propriété intellectuelle}
\label{sec:Gestion et propriété intellectuelle}

\section{Valorisation et Diffusion}
\label{sec:Valorisation et Diffusion}
\subsection{Plan de valorisation} 

  

\subsubsection{Option 1 : Création d'une spin-off universitaire (commercialisation de RedactoSci)} 

Cette approche vise la création de valeur économique et le transfert de technologie vers le marché. 

  

\textbf{Avantages :} 

\begin{itemize} 

    \item Génération de revenus : Potentiel de profits pour l'université, les chercheurs impliqués et la spin-off elle-même. 

    \item Développement dédié ou une meilleure focalisation stratégique : La spin-off, en tant qu'entité indépendante, peut se concentrer pleinement sur le développement, la maintenance et la commercialisation de l'outil, avec une flexibilité supérieure aux structures académiques. 

    \item Impact économique : Contribution au développement économique local (emplois, innovation). 

    \item Financement externe ou une opportunité de financement facilitée : Facilite l'attraction d'investissements privés (capital-risque) pour accélérer la croissance. 

\end{itemize} 

  

\textbf{Inconvénients :} 

\begin{itemize} 

    \item Barrières à l'accès : Le coût peut limiter l'utilisation de l'outil, réduisant son adoption large par la communauté scientifique ou le grand public. 

    \item Risque entrepreneurial : La plupart des spin-offs universitaires ne rencontrent pas un succès financier immédiat et comportent des risques d'échec. 

    \item Conflit d'intérêts : Peut créer des tensions entre la culture universitaire (partage des connaissances) et la culture d'entreprise (confidentialité, profit). 

\end{itemize} 

  

\subsubsection{Option 2 : Offrir gratuitement comme service public (attractivité et impact)} 

Cette approche privilégie l'impact sociétal, la visibilité et le rayonnement académique. 

  

\textbf{Avantages :} 

\begin{itemize} 

    \item Attraction des talents : Un outil de pointe en libre accès augmente la visibilité du laboratoire/département, attirant les meilleurs étudiants, chercheurs et partenaires académiques. 

    \item Impact sociétal maximal : Favorise l'adoption généralisée de l'outil, maximisant son influence sur la recherche, l'éducation ou la société en général. 

    \item Renforcement de la réputation : Positionne l'université comme un acteur clé de l'innovation ouverte et du service public. 

    \item Accès aux financements publics : Peut faciliter l'obtention de subventions de recherche axées sur l'intérêt général et la collaboration. 

\end{itemize} 

  

\textbf{Inconvénients :} 

\begin{itemize} 

    \item Coûts de maintenance/développement : Le financement continu de la maintenance, des mises à jour et de l'infrastructure peut être difficile à assurer sans revenus directs. 

    \item Manque d'agilité commerciale : Les structures publiques peuvent manquer de la flexibilité nécessaire pour répondre rapidement aux demandes du marché ou aux besoins spécifiques des utilisateurs professionnels. 

    \item Difficulté à capter la valeur économique : La valeur générée profite indirectement à la société ou à d'autres entreprises, sans retour financier direct pour les créateurs ou l'université. 

\end{itemize} 

  

\subsubsection{Plan de Valorisation Recommandé : Un Modèle Hybride} 

Le plan le plus efficace combine souvent les avantages des deux approches : 

  

\begin{itemize} 

    \item \textbf{Lancement en Accès Limité/Freemium :} Offrir une version de base de l'outil gratuitement à la communauté académique pour maximiser l'adoption et l'attraction des talents. 

    \item \textbf{Création d'une Spin-off pour les Services Premium :} La spin-off pourrait vendre des versions améliorées, des fonctionnalités avancées, un support technique dédié, ou des services de conseil aux entreprises (licences commerciales). 

    \item \textbf{Stratégie de Propriété Intellectuelle (PI) Claire :} Protéger l'innovation par des brevets pour permettre la commercialisation par la spin-off, tout en prévoyant des licences spécifiques pour la recherche académique. 

\end{itemize} 

  

Ce modèle permet de maintenir l'impact académique tout en explorant le potentiel commercial, assurant ainsi la viabilité à long terme de l'outil et de ses développements. 

  

\subsection{Plan de diffusion} 

  

Identifier deux publics : 

  

\subsubsection{Pour les éditeurs scientifiques} 

\begin{itemize} 

    \item Adaptez l'article à la revue visée : Choisissez des revues (comme IEEE ou Springer) dont la portée correspond à votre domaine de recherche, en respectant leurs critères et formats spécifiques. 

    \item Mettre en avant l'innovation : Soulignez le caractère novateur de RedactoSci et son impact potentiel sur le domaine scientifique. 

    \item Utilisez un langage scientifique précis : Présentez vos résultats de manière claire, factuelle et rigoureuse. 

    \item Mettre en valeur les données : Fournissez des données solides pour soutenir vos conclusions, en démontrant la méthodologie de recherche. 

\end{itemize} 

  

\subsubsection{Pour les comités d’éthique de la recherche} 

\begin{itemize} 

    \item Démontrez le respect des principes éthiques : Prouvez que votre projet respecte les normes éthiques, notamment le consentement éclairé, la confidentialité et la protection des données. 

    \item Décrire les procédures en détail : Expliquez clairement comment vous allez obtenir le consentement des participants, la durée prévue de leur implication et les mesures prises pour minimiser les risques potentiels. 

    \item Proposez des documents de soutien : Incluez tous les documents nécessaires, tels que les formulaires de consentement, les protocoles de recherche et les procédures de protection des données. 

    \item Mettre en avant les bénéfices pour la société : Expliquez comment votre recherche bénéficiera à la société, que ce soit en termes de connaissances, d'amélioration des soins ou de développement de nouvelles technologies. 

\end{itemize} 



\section{Contributions individuelles}
\label{sec:Contributions individuelles}

\newpage
% 5) Bibliographie via BibTeX


\newpage
\pagenumbering{roman}
\appendix



\end{document}
