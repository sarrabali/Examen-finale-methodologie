\documentclass{beamer}
\usetheme{Madrid}
\setbeamertemplate{bibliography item}{\insertbiblabel}

\usepackage[utf8]{inputenc}
\usepackage[T1]{fontenc}
\usepackage[french]{babel}
\usepackage{lmodern}
\usepackage{amsmath, amssymb}
\usepackage{graphicx}
\usepackage{booktabs}
\usepackage{geometry}
\usepackage{csquotes}
\usepackage{tabularx}

\graphicspath{{figures/}}

\usepackage{hyperref}

% --- MÉTADONNÉES DU DOCUMENT ---
\title[RedactoSci]{De la méthode à l’intégrité : gouvernance éthique et légale d’un outil d’IA
générative pour la recherche « RedactoSci »}

\author[UQO INF5163 Examen Finale- groupe 8]{Paulin Rodrigue Njayou Tchapda, Sarra Yasmine Bali,
	Rachidatou Mabey Insa et Fatimata Zahra Diop \\[1em]
	\emph{Université du Québec en Outaouais}}

\date{9 Décembre 2025, Automne 2025}
\emergencystretch=1em
\begin{document}

% ------------------ TITRE ------------------
\begin{frame}
    \titlepage
\end{frame}

% ------------------ PLAN ------------------
\begin{frame}{Plan de la présentation}
\tableofcontents
\end{frame}

% ============================================================
\section{Introduction}
% ============================================================

\begin{frame}{Introduction et justification du sujet}
\begin{itemize}
    \item Expansion fulgurante de l’IA générative en 2025 : texte, images, code, audio.
    \item Les LLM permettent de produire en quelques secondes des contenus complexes \cite{Humlund2025}.
    \item Les milieux universitaires sont transformés : rédaction, analyse, recherche documentaire.
    \item Notre équipe propose \textbf{RedactoSci}, un modèle spécialisé formé sur millions d’articles.
\end{itemize}

\vspace{0.3cm}
\textbf{Mais :} intégrité, plagiat, hallucinations, propriété intellectuelle et gouvernance deviennent critiques \cite{UQAM2025}.
\end{frame}

% ============================================================
\section{Contexte et problématique}
% ============================================================

\begin{frame}{Impact de l’IA générative sur la recherche}
\begin{itemize}
    \item Automatisation croissante des tâches rédactionnelles.
    \item Forte dépendance possible aux outils d’IA.
    \item Risques : erreurs factuelles, dilution de l’analyse humaine \cite{UQAM2025}.
    \item Besoin de mécanismes institutionnels pour garantir une utilisation responsable.
\end{itemize}
\end{frame}

\begin{frame}{Problématique}
Les chercheurs font face à :
\begin{itemize}
    \item surcharge informationnelle \cite{Humlund2025}.
    \item délais serrés,
    \item syndrome de la page blanche,
    \item maintien difficile de l’intégrité scientifique.
\end{itemize}

\textbf{RedactoSci a été conçu pour répondre à ces défis.}
\end{frame}

% ============================================================
\section{Objectifs}
% ============================================================

\begin{frame}{Objectifs SMART de RedactoSci}
\begin{itemize}
    \item \textbf{S} : assister la rédaction scientifique.
    \item \textbf{M} : réduire de 30\% le temps de production.
    \item \textbf{A} : automatiser tâches : plans, résumés, reformulation, code.
    \item \textbf{R} : faciliter l'accès aux ressources fiables.
    \item \textbf{T} : déploiement sur un semestre universitaire.
\end{itemize}
\end{frame}

% ============================================================
\section{Méthodologie (Rappel)}
% ============================================================

\begin{frame}{Rappel du schéma méthodologique}
\begin{itemize}
    \item Collecte de données scientifiques : articles, rapports, thèses.
    \item Nettoyage : suppression doublons, correction erreurs, normalisation.
    \item Modèle GPT-like basé sur Transformer \cite{antonesi2025review}.
    \item Fine-tuning : résumés, concepts, reformulations, suggestions bibliographiques.
    \item Évaluations intermédiaires + retours utilisateurs.
\end{itemize}
\end{frame}

% ============================================================
\section{Analyse : intégrité, éthique, ÉDI}
% ============================================================

\subsection{Analyse éthique}
\begin{frame}{Analyse d’intégrité et d’éthique}
Trois enjeux majeurs :
\begin{enumerate}
    \item \textbf{Plagiat algorithmique} : reproduction involontaire de données d'entraînement \cite{carlini2021extracting}.
    \item \textbf{Hallucinations} : production d’informations fausses \cite{huang2025survey}.
    \item \textbf{Perte de compétence} : dépendance excessive et diminution des capacités critiques \cite{gerlich2025ai}.
\end{enumerate}
\end{frame}

\begin{frame}{Plagiat algorithmique}
\begin{itemize}
    \item Les LLM peuvent mémoriser et regénérer du contenu protégé.
    \item \textbf{Solutions RedactoSci :}
    \begin{itemize}
        \item reformulation automatique,
        \item détection des formulations proches,
        \item rappel des obligations de citation.
    \end{itemize}
\end{itemize}
\end{frame}

\begin{frame}{Fabrication de données (Hallucinations)}
\begin{itemize}
    \item Contenu plausible mais faux ou inventé.
    \item \textbf{Solutions :}
    \begin{itemize}
        \item système de détection d’incertitude,
        \item avertissements,
        \item encouragement à vérification manuelle.
    \end{itemize}
\end{itemize}
\end{frame}

\begin{frame}{Perte de compétence (Deskilling)}
\begin{itemize}
    \item Risque de réduire la pensée critique et l’analyse autonome.
    \item \textbf{RedactoSci vise à :}
    \begin{itemize}
        \item renforcer l’apprentissage actif,
        \item fournir explications méthodologiques,
        \item éviter remplacement du raisonnement humain.
    \end{itemize}
\end{itemize}
\end{frame}

\subsection{Aspects ÉDI}
 \begin{frame}{Aspects ÉDI}
	 \begin{table}[h!] \centering \small \renewcommand{\arraystretch}{1.3} \setlength{\tabcolsep}{6pt}
		 \begin{tabularx}{\textwidth}{|>{\raggedright\arraybackslash}X|>{\raggedright\arraybackslash}X|} \hline \textbf{Aspect} &\textbf{Description et rôle de \emph{RedactoSci}} \\ \hline \textbf{Biais linguistiques}& La recherche se fait avec la language anglais majoritairement. \emph{RedactoSci} intègre des données multilingues pour favoriser la diversité scientifique. \\ \hline \textbf{Biais culturels et de genre}& Le langage scientifique peut reproduire des stéréotypes. \emph{RedactoSci} ajuste les formulations fournies pour assurer le respect et l'inclusion. \\ \hline \textbf{Transparence et inclusion dans la conception}& Les données sont documentées et traçables. \emph{RedactoSci} assure la fiabilité des données et la transparence. \\ \hline \end{tabularx} \end{table} \end{frame}

% ============================================================
\section{Propriété intellectuelle}
% ============================================================

\begin{frame}{Gestion de la propriété intellectuelle}
\begin{itemize}
    \item L’IA n’est pas considérée comme auteur légal.
    \item L’utilisateur humain demeure responsable du contenu.
    \item Obligation de divulguer l’usage de l’IA.
    \item Protection des données : éviter prompts sensibles, anonymisation.
\end{itemize}
\end{frame}

\begin{frame}{Documentation et partage}
\begin{itemize}
    \item \textbf{Science ouverte :} transparence, DOI, code public, model card.
    \item \textbf{Secret commercial :} API, NDA, modules propriétaires.
    \item \textbf{Solution recommandée : modèle hybride} :
    \begin{itemize}
        \item version gratuite = fonctions essentielles,
        \item version payante = fine-tuning, analyses avancées.
    \end{itemize}
\end{itemize}
\end{frame}

% ============================================================
\section{Valorisation et diffusion}
% ============================================================

\begin{frame}{Plan de valorisation}
Trois options :
\begin{itemize}
    \item création d’une spin-off universitaire,
    \item service gratuit pour maximiser l’impact scientifique,
    \item \textbf{modèle hybride} (recommandé).
\end{itemize}
\end{frame}

\begin{frame}{Diffusion aux éditeurs scientifiques}
\begin{itemize}
    \item Respect normes IEEE et Springer.
    \item Transparence sur l’usage de l’IA.
    \item Résultats reproductibles et vérifiables.
\end{itemize}
\end{frame}

\begin{frame}{Diffusion aux comités d’éthique}
\begin{itemize}
    \item Démarche conforme à protection des données.
    \item Fourniture des protocoles et documents requis.
    \item Mise en évidence des bénéfices pédagogiques.
\end{itemize}
\end{frame}

% ============================================================
\section{Conclusion}
% ============================================================

\begin{frame}{Conclusion}
\begin{itemize}
    \item RedactoSci répond à des défis méthodologiques majeurs.
    \item Importance d’une gouvernance éthique rigoureuse.
    \item Le modèle hybride concilie accessibilité et durabilité.
    \item L’IA peut soutenir la recherche si elle est utilisée de manière responsable.
\end{itemize}
\end{frame}

% ============================================================

\begin{frame}[allowframebreaks]{Références}
    \bibliographystyle{ieeetr}  % STYLE IEEE
    \bibliography{references}   % FICHIER .bib
\end{frame}
\end{document}